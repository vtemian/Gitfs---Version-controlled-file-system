
\chapter{Conclusion}

\label{ch:conclusions}

\section{Summary of Thesis Achievements}

Since the purpose of this thesis was to build a tool which will help non-technical users to create versioned content without having knowledge about any version control system, it is considered that its purpose has been achieved.

Gitfs is an elegant solution to this problem, offering a natural way to view and edit any type of content (from text to photos or music) using the file system interface and hiding the entire complex versioning process in an easy to understand concurrency model.

In order to create and maintain Gitfs, multiple contributions, to already mentioned libraries were made. Beside those, tools were created to automate releases and binaries publications.

\section{Applications}

Gitfs is a functional product, already used by Presslabs for almost two years, where more than 200 clients use it to manage their website code. Since its original launch, in October 2014, 15 other releases were made.

Besides Presslabs, other people have showed interest for Gitfs. On Github, 1025 people stared the repository and 63 people have forked it in order to make contributions. From those numbers it is very hard to tell how many are using it, but having 11 direct contributions and around 20 indirect contributions through bug reports, it is considered that Gitfs is doing what it was meant to do and that is helping people.

This file system was presented in multiple countries, within different conferences (local Timisoara meetups, Python conference in Iasi 2014, PyCon Sweeden 2015, PyCon Italy 2015, OpenTech in London 2015) with the biggest one being EuroPython 2015.

\section{Future Work}

Even if the project is mature, a lot of improvements can be made:
\begin{itemize}
    \item Refactor concurrency model to use a MPI (Message Passing Interface) in order to reduce the locks needed.
    \item Decouple the synchronization mechanism from the file system part.
    \item Reuse the synchronization mechanism with different frontend (instead of file system).
    \item Follow multiple branches and remotes.
    \item Allow users to control when a commit, push or fetch is made.
    \item Allow users to change file system options at runtime, using configuration files.
    \item Add support for LFS (git's large file storage), in order to offer a better support for big files.
    \item Improve commit's message. Right now, when multiple files are changed or created, a single message showing how many files were affected is created.
    \item Add support for multiple platforms. Right now, Gitfs has official binaries only for Ubuntu and other people have created packages for Fedora, ArchLinux and MacOS.
\end{itemize}
    