\chapter{Concurrency Model}

\label{ch:concurrencymodel}

\section{Merging}
\section{Options}
In order to use options when mounting gitfs, you need to append the options as an argument when using the mount command like this: -o option1=value1,option2=value2,option3=value3...

\begin{itemize}
    \item remote\_url: the URL of the remote
    \item branch: the branch name to follow (default: master)
    \item repo\_path: the location where the repositories will be cloned
    (default: $/var/lib/gitfs/repo_path$)
    \item max\_size: the maximum file size in MBs allowed for an individual file. If set to 0, then allow any file size (default: 10MB)
    \item user: the user that will mount the file system (default: root)
    \item group: the group that will mount the file system (default: root)
    \item committer\_name: the name that will be displayed for all the commits (default: user)
    \item committer\_email: the email that will be displayed for all the commits (default: user@FQDN)
    \item merge\_timeout:	the interval between idle state and commits/pushes (default: 5s)
    \item fetch\_timeout: 	the interval between fetches (default: 30s)
    \item log: the path of the log file. Special name syslog will log to the system logger (default: syslog)
    \item log\_level: the logging level. One of error, warning, info, debug (default: warning)
    \item debug: he switch that sets the log level to debug and also enables FUSE’s debug (default: false)
    \item username: the username for HTTP basic auth 
    \item password: the password for HTTP basic auth
    \item key:the path of the SSH private key. NOTE: the public key is constructed by appending .pub to this path and the file MUST exist (default: \textdollar HOME/.ssh/id\_rsa)
\end{itemize}